\documentclass{beamer}
\usetheme{Frankfurt}
\usepackage[dutch]{babel}
\usecolortheme{default}

\title{Elektrotechniek basis}
\author{ir. BEng. Job Meijer}
\institute{Club van Draadje}


\AtBeginSection[]
{
	\begin{frame}<beamer>
		\frametitle{Komend onderwerp}
		\tableofcontents[currentsection]
	\end{frame}
}

\begin{document}

% Welkom
\begin{frame}
    \titlepage
\end{frame}

% inhoudsopgave
\begin{frame}{Inhoud}
	\tableofcontents
\end{frame}

% Inhoud presentatie

\section{Wat is elektriciteit?}

\begin{frame}{Atoom}
	Atoom \\
\end{frame}

\begin{frame}{Elektrische lading}
	lading \\
\end{frame}

\begin{frame}{Elektrische stroom}
	stroom\\
\end{frame}

\begin{frame}{Elektrische spanning}
	spanning \\
\end{frame}

\begin{frame}{Elektrische weerstand}
	weerstand \\
\end{frame}

\begin{frame}{Elektromagnetisme}
	magneet \\
\end{frame}

\begin{frame}{Elektrisch veld}
	veldje \\
\end{frame}
	
	
\section{Elektriciteit en energie}
\begin{frame}{Elektriciteit en energie}
	hoi dit is elektrische energie. \\
\end{frame}



\section{Elektrisch netwerk}
		\begin{block}{Math equations}
			The Pythagorean theorem is a fundamental relation in Euclidean geometry among the three sides of a right triangle:
			\pause 
			\begin{equation}
				a^{2}+b^{2}=c^{2}
			\end{equation}  
		\end{block}   



\section*{Referenties}	


	
	



% force table of content entries
\begin{frame}	
\end{frame}


\end{document}